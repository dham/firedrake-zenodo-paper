\documentclass[a4paper,11pt]{article}

\usepackage[margin=2.5cm]{geometry}
\usepackage{palatino}

\usepackage[doi=true,url=true,style=authoryear,dashed=false,firstinits=true,maxnames=2,maxbibnames=99,backend=biber]{biblatex}
\renewbibmacro{in:}{%
  \ifentrytype{article}{}{%
  \printtext{\bibstring{in}\intitlepunct}}}

\bibliography{bibliography}

\begin{document}

\title{Provenance tracking for scientific software toolchains through on-demand release and archiving}

\author{David A. Ham\\Department of Mathematics, Imperial College London}

\maketitle

\section*{Abstract}

\section{Introduction}

There is an emerging consensus that published computational science results
must be backed by a provenance trail tying results to the exact versions of
input data and the code which generated them. There is also now an
impressive range of web services devoted to revision control of software,
and the archiving in citable form of both software and input data. 
However, it is also observably the case that very many papers are published
which do not adhere to these high standards of transparency and
reproducibility. 

The simplest computational workflows are well-supported by the range of
currently available tools. If a scientist has a data set, and writes some
code which acts on that data set then he or she can host the code on an open
platform such as GitHub or BitBucket, and archive both the code and the data
on a platform such as Zenodo or Figshare. For completeness, the scientist
should preferably also list the version of the comilers and/or interpreters
employed to run the code. Thanks in large part to the high quality of
software engineering and interface design on behalf of the web services,
this workflow enables a scientist to achieve best practice for very little
effort. 



\section{Provenance vs credit}

\end{document}
